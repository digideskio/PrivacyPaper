\documentclass[twocolumn, a4paper, 10pt]{article}
\usepackage[cmex10]{amsmath}

% makes everything a bit tighter
\usepackage{microtype}

\usepackage{amsopn}
\usepackage{amsthm}
\usepackage{amsmath}

\usepackage{hyperref}

\usepackage{graphicx}
\graphicspath{{./figures/}}
\DeclareGraphicsExtensions{.pdf,.jpeg,.png,.eps, .svg}

% if you want to draw sth: have a look at tikz
\usepackage{tikz}
\usetikzlibrary{positioning}
\usetikzlibrary{calc, fit, shapes, decorations.markings, calendar}

% comments for yourself
\newcommand{\me}[1]{{\color{red}#1}}
% and on the margin
\newcommand{\meb}[1]{\marginpar{\small\textcolor{red}{#1}}}

% comments for benny
\newcommand{\cb}[1]{{\color{red}#1}}
% and on the margin
\newcommand{\cbb}[1]{\marginpar{\small\textcolor{red}{#1}}}

% for lorem ipsum - you can remove that
\usepackage{lipsum}

\begin{document}
\title{
    {\Large An Introduction to Location Privacy \\}
    {\large (Insert cool subtitle)}
}

\author{
    Simon Kalt
}

\maketitle

\def\abstractname{{\textbf Abstract}}
\begin{abstract}
{
% \bfseries
Devices that collect and use location data are practically ubiquitous in our society. Since the location data collected by these devices can be used to infer highly private information about their users, a need for privacy of location data has arisen. This paper aims to introduce the reader to the concept of location privacy and the developments over the recent years in this field of study.
}
\end{abstract}


\section{Introduction}
\lipsum[1-3]


% Bibliography

{
    \bibliographystyle{plain}
    \bibliography{bibliography,privacy}
}

\appendix
\end{document}
