\documentclass[twocolumn, a4paper, 10pt]{article}
\usepackage[cmex10]{amsmath}

% makes everything a bit tighter
\usepackage{microtype}

\usepackage{amsopn}
\usepackage{amsthm}
\usepackage{amsmath}

\usepackage{hyperref}

\usepackage{graphicx}
\graphicspath{{./figures/}}
\DeclareGraphicsExtensions{.pdf,.jpeg,.png,.eps, .svg}

% if you want to draw sth: have a look at tikz
\usepackage{tikz}
\usetikzlibrary{positioning}
\usetikzlibrary{calc, fit, shapes, decorations.markings, calendar}

% comments for yourself
\newcommand{\me}[1]{{\color{red}#1}}
% and on the margin
\newcommand{\meb}[1]{\marginpar{\small\textcolor{red}{#1}}}

% comments for benny
\newcommand{\cb}[1]{{\color{red}#1}}
% and on the margin
\newcommand{\cbb}[1]{\marginpar{\small\textcolor{red}{#1}}}

% for lorem ipsum - you can remove that
\usepackage{lipsum}

\begin{document}
\title{\Large Your Thesis Topic}

\author{
	Simon Kalt
}

\maketitle

\def\abstractname{{\textbf Abstract}}
\begin{abstract}
{
\bfseries
 This section should contain an abstract of your paper. The abstract's purpose is to report what your paper is about, and why the reader should bother to read the rest of it. It should be succinct and precise, but will usually also include a few sentences of motivation.
}
\end{abstract}


\section{Introduction}
\lipsum[1-3]


% Bibliography

{
	\bibliographystyle{plain}
	\bibliography{bibliography,privacy}
}

\appendix
\end{document}
